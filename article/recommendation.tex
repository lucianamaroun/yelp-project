\section{Recommendation of Reviewers}
Collaborative filtering is an algorithm which combines the evaluation of
users in a bunch of topics in order to predict users rating in unseen ones. In
this process, the evaluated elements have a set of attributes that define them
and are closely related to their scores. The users, on the other hand, have a
representative vector which are coefficients for the attributes and informes the
importance given for each one of them. Considering this model, it is possible to
discover close friends, which does not necessarily have to be friends in Yelp,
by filtering those with small euclidian distance.

The algorithm basically solves an optmization problem which consists of:

\[
min_{x^1,...,x^{n_e},\theta^1,...\theta^{n_u}}
  \sum_{(i,j):r(i,j)=1} ((\theta^j)^T x^i - y^{(i,j)})^2 +
  \lambda \sum_{i=1}^{n_e} \sum_{k=1}^{n} (x_k^i)^2 +
  \lambda \sum_{j=1}^{n_u} \sum_{k=1}^{n} (\theta_k^j)^2
\]

In this formula, $x^i$ is the vector of attributes for the element $i$,
$\theta^j$ is the coefficient vector for user $j$, $r(i,j)$ is $1$ if user $j$
rated element $i$ and $0$ otherwise, $y^{(i,j)}$ is the rating given by user $j$
to element $i$, $n_e$ is the number of evaluated elements, $n_u$ is the number
of users, $n$ is the number of attributes, i.e., the dimension of $x$ and
$\theta$ and $\lambda$ is a regularization factor. This expression tries to find
the set of $x^i$, $i=1,...,n_e$, and $\theta^j$, $j=1,...,n_u$ which dot product
$\theta^j \cdot x^i$ approximates the existing ratings $y^{(i,j)}$. The last two
terms simply avoids the increase of the calculated parameters in module.

In the following, a clustering algorithm is performed on the set of obtained
$\theta^j$, $j=1,...,n_u$. Then, we recommend reviewers from the same cluster.
